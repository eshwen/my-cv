\cvsection{Education}

\newcommand{\vpaddingEduNorm}{\vspace{1mm}} % used to create more vertical space between entries in cvitems environment
\newcommand{\vpaddingEduSubpoint}{\vspace{0.75mm}} % used to create more vertical space between entries in an itemize environment within the cvitems environment


\begin{cventries}
    \cventry
        {Doctor of Philosophy in Physics} % Degree
        {University of Bristol} % Institution
        {Bristol, United Kingdom} % Location
        {Sep. 2016 -- Present} % Date(s)
        {
        \begin{cvitems} % Description(s) bullet points
            \item {Thesis title: \textbf{Hadronic Dark Matter Searches at CMS at 13 TeV} --- \entrydatestyle{Under supervision of H. Fl\"{a}cher. Expected submission in April 2020}}
            \vpaddingEduSubpoint
            \begin{itemize}
                \item[\bullet]{Performing searches for dark matter arising from various physics models by analysing data from LHC's CMS experiment.}
                \vpaddingEduSubpoint
                \item[\bullet]{Simulated data for signal and background processes compared to LHC data in a statistical analysis framework to prove or disprove a model.}
                \vpaddingEduSubpoint
                \item[\bullet]{Code to simulate data and perform each step of analysis predominantly written in Python, leveraging modern data science tools and distributed computing where possible to process multiple terabytes of data.}
            \end{itemize}
            \vpaddingEduNorm
            % ----------
            \item {\textbf{Postgraduate student representative} for the particle physics group, 2019--20 --- \entrydatestyle{Role in the Student-Staff Liaison Committee for the School of Physics}}
            \vpaddingEduNorm
            % ----------
            \item {\textbf{CERN} (European Organisation for Nuclear Research), Geneva, Switzerland --- \entrydatestyle{Long term attachment, Oct. 2017 -- Mar. 2019}}
            \vpaddingEduNorm
            % ----------
            \item {\textbf{Calorimeter Layer-2 on call expert} and \textbf{Level-1 Trigger shifter}, Geneva, Switzerland --- \entrydatestyle{Additional responsibilities at CERN}}
            \vpaddingEduSubpoint
            \begin{itemize}
                \item[\bullet]{Trained to carry out tasks to assist CMS experiment so operations continue smoothly.}
                \vpaddingEduSubpoint
                \item[\bullet]{Helped develop and implement software for subsystem of Level-1 Trigger to apply corrections and calibrations to data.}
            \end{itemize}
        \end{cvitems}
        }

    \cventry
        {Master of Physics with Honours in Physics with Astrophysics. Award: First Class (77\,\%)}
        {University of Exeter}
        {Exeter, United Kingdom}
        {Sep. 2012 -- Jul. 2016}
        {
        \begin{cvitems} % Description(s) bullet points
            \item {Dissertation title: \textbf{Simulations of Exoplanet Light Curves} --- \entrydatestyle{Under supervision of T. Harries}}
            \vpaddingEduSubpoint
            \begin{itemize}
                \item[\bullet]{Developed software in C to simulate how photons interact in simple planetary atmospheres, producing light curves akin to data from telescopes. Maps of density and composition were also created for model planets.}
                \vpaddingEduSubpoint
                \item[\bullet]{More complex atmospheres can also be modelled and compared to real exoplanets to infer their composition.}
                \vpaddingEduSubpoint
                \item[\bullet]{Utilised Monte Carlo random sampling for scattering of photons, and parallelisation to efficiently run the code over millions of photons.}
            \end{itemize}
        \end{cvitems}
        } % Consider adding relevant/highlighted modules with highest marks - Relativity and Cosmology (92 %), Mathematics for Physicists (91 %), Planets and their Atmospheres (87 %), Scientific Programming in C (71 %)

    \cventry
        {Secondary school qualifications}
        {Monmouth Comprehensive School}
        {Monmouth, United Kingdom}
        {Sep. 2005 -- Aug. 2012}
        {
        \begin{cvitems}
            \item {\textbf{A Level}, 2010--12 --- \entrydatestyle{Biology (A*), Mathematics (A*), Physics (B), Chemistry (AS Level) (B)}}
            \vpaddingEduNorm
            % ----------
            \item {\textbf{Open University}, 2011--12 --- \entrydatestyle{Introducing Astronomy (10 credit course)}}
            \vpaddingEduNorm
            % ----------
            \item {\textbf{GCSE}, 2008--10 --- \entrydatestyle{10 including English Language and Mathematics at grades A* (4) to A (6)}}
            %\vpaddingEduNorm
            % ----------
            %\item {\textbf{WJEC Key Skill}, 2008-10 --- \entrydatestyle{Communication (Level 2)}}
        \end{cvitems}
        }
\end{cventries}
