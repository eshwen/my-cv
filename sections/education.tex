\cvsection{Education}

\newcommand{\vpaddingEduNorm}{\vspace{1mm}} % used to create more vertical space between entries in cvitems environment
\newcommand{\vpaddingEduSubpoint}{\vspace{0.75mm}} % used to create more vertical space between entries in an itemize environment within the cvitems environment


\begin{cventries}
    \cventry
        {Doctor of Philosophy in Physics} % Degree
        {University of Bristol} % Institution
        {Bristol, United Kingdom} % Location
        {2016 -- Present} % Date(s)
        {
        \begin{cvitems} % Description(s) bullet points
            \item {Thesis title: \textbf{Searches for dark matter with a focus on invisibly decaying Higgs bosons in the ttH, VH, and ggH channels using the full Run-2 dataset from the CMS experiment} --- \entrydatestyle{Under supervision of H. Fl\"{a}cher. Expected submission in July 2020}}
            \vpaddingEduSubpoint
            \begin{itemize}
                \item[\bullet]{Explored various physics models in search of dark matter by analysing data from LHC's CMS experiment. Set \textbf{world leading limits} on Higgs boson decay to invisible states.}
                \vpaddingEduSubpoint
                \item[\bullet]{Executed robust, comprehensive statistical analysis on real and simulated data, meticulously documenting concepts, results and code.}
                \vpaddingEduSubpoint
                \item[\bullet]{Composed versatile, robust, efficient code to simulate data and perform each step of analysis. Written predominantly in Python, leveraging modern data science tools, vectorisation, and distributed computing to process multiple terabytes of data.}
            \end{itemize}
            \vpaddingEduNorm
            % ----------
            \item {\textbf{Postgraduate student representative} for the particle physics group, 2019--20 --- \entrydatestyle{Role in Student-Staff Liaison Committee for School of Physics}}
            \vpaddingEduNorm
            % ----------
            \item {Long term attachment at \textbf{CERN}, the world's largest particle physics laboratory --- \entrydatestyle{18 month placement abroad in Geneva, Switzerland}}
            \vpaddingEduNorm
            % ----------
            \item {\textbf{Calorimeter Layer-2 on call expert} and \textbf{Level-1 Trigger shifter}, Geneva, Switzerland --- \entrydatestyle{Additional responsibilities at CERN}}
            \vpaddingEduSubpoint
            \begin{itemize}
                \item[\bullet]{Assisted in detector operations and monitoring so CMS experiment could take data efficiently and operate smoothly.}
                \vpaddingEduSubpoint
                \item[\bullet]{Developed and deployed software for subsystem of Level-1 Trigger to apply corrections and calibrations to data on the fly.}
            \end{itemize}
        \end{cvitems}
        }

    \cventry
        {Master of Physics with Honours in Physics with Astrophysics. Award: First Class}  % 4.0 GPA equivalent. 77\,\%
        {University of Exeter}
        {Exeter, United Kingdom}
        {2012 -- 2016}
        {
        \begin{cvitems} % Description(s) bullet points
            \item {Dissertation title: \textbf{Simulations of Exoplanet Light Curves} --- \entrydatestyle{Under supervision of T. Harries}}
            \vpaddingEduSubpoint
            \begin{itemize}
                \item[\bullet]{Developed software in C to simulate photons interacting simple planetary atmospheres, producing light curves akin to data from telescopes. Visualised model planets with maps of density and composition.}
                \vpaddingEduSubpoint
                \item[\bullet]{Support for modelling more complex atmospheres for comparison to real exoplanets to infer their composition.}
                \vpaddingEduSubpoint
                \item[\bullet]{Utilised Monte Carlo random sampling for scattering of photons, and parallelisation to efficiently run the code over millions of them.}
            \end{itemize}
        \end{cvitems}
        } % Consider adding relevant/highlighted modules with highest marks - Relativity and Cosmology (92 %), Mathematics for Physicists (91 %), Planets and their Atmospheres (87 %), Scientific Programming in C (71 %)

    \cventry
        {Secondary school qualifications}
        {Monmouth Comprehensive School}
        {Monmouth, United Kingdom}
        {2005 -- 2012}
        {
        \begin{cvitems}
            \item {\textbf{A Level}, 2010--12 --- \entrydatestyle{Biology (A*), Mathematics (A*), Physics (B), Chemistry (AS Level) (B)}}
            \vpaddingEduNorm
            % ----------
            %\item {\textbf{Open University}, 2011--12 --- \entrydatestyle{Introducing Astronomy (10 credit course)}}
            %\vpaddingEduNorm
            % ----------
            \item {\textbf{GCSE}, 2008--10 --- \entrydatestyle{10 including English Language and Mathematics at grades A* (4) to A (6)}}
            %\vpaddingEduNorm
            % ----------
            %\item {\textbf{WJEC Key Skill}, 2008-10 --- \entrydatestyle{Communication (Level 2)}}
        \end{cvitems}
        }
\end{cventries}
