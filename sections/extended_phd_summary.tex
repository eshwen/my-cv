\cvsection{Extended PhD summary}

\begin{cvparagraph}
    Dark matter is a pervasive, yet largely unknown, presence in the universe. It can only be observed indirectly through its gravitational influence---seeding galaxies and providing enough bulk to prevent stars escaping, or by warping the very fabric of space-time to smear the light that passes near it. There are many properties which may be uncovered by astrophysics, but particle physics contains a swathe of new avenues and approaches to hopefully pinpoint the origins of dark matter.

    The Higgs boson has caught the attention of the high energy physics community, and even the public eye, like no other particle in recent memory. Its rather recent discovery in 2012 still leaves many of its characteristics unknown, or lacking the precision of the other fundamental particles. The instability of this particle, like many of its heavy counterparts, forces it to decay into lighter ones. Dark matter particles may be on one of the branches. The Large Hadron Collider (LHC) is able to produce Higgs bosons in abundance by colliding protons at extraordinary energies. I aim to measure the frequency with which it decays to particles that cannot be recorded by our detector, labelled ``invisible'', like neutrinos. If there is a disparity between the frequency I obtain and the prediction given by the Standard Model of particle physics (which does not account for dark matter), it lends credence to Higgs boson unlocking the secrets of how dark matter can be created.

    Over ten thousand trillion proton-proton collisions have occurred at the LHC: a lot of data! Despite the wealth of new insights it may yield, working with big data in collaboration with thousands of others presents its own challenges. In my smaller working group, we have developed a lot of code to perform all aspects of an analysis in search of new physics: accessing and locally storing data from the LHC, interfacing with tools developed by other colleagues, generating simulated samples of expected signal and background to compare to the data, and build up the remaining components of the analysis. Perhaps the worst bottleneck we face is the ability to run the analysis over all data and simulation efficiently. Measures are taken to reduce the number of events we have to process, though it still leaves us with the products of tens of billions of collisions to sift through. With vectorisation and batch computing, we can accomplish this in as little as 30 minutes; a time frame that's essential considering the number of iterations and the development required to create a complete analysis. The use of industry-standard tools such as \texttt{numpy}, \texttt{pandas}, and \texttt{matplotlib} also improve efficiency, bestow a range of useful features, and provide continued support and improvements.

    I was fortunate enough to be based at CERN, just outside Geneva, for eighteen months of my PhD. In addition to better collaboration with colleagues stationed there---and experiencing the atmosphere of the world's hub for particle physics---I was able to undertake more responsibility. Working shifts to maintain subsystems of the detector, and being an on-call expert for another subsystem were some of those.

    In summary, my main research topic is a search for dark matter via invisible decays of the Higgs boson. I have additionally held several detector-related responsibilities and undergraduate teaching duties, sometimes simultaneously. In all cases, direct collaboration with a small group of people, but wider involvement within my experiment as a whole have been important. The development of clear, efficient code to execute many of these tasks is paramount to ensure smooth and successful operation. Being based abroad, while presenting new challenges, also gave more opportunities for personal and professional development.

    % Look over thesis writing circle extract as well for more inspiration

    % Not sure if they want a broad overview of what I've achieved during my PhD, or focus on specifically my main research topic.
\end{cvparagraph}
